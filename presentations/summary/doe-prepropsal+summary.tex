\documentclass[11pt,final]{article}
\usepackage{fullpage} % sets to default of 1 inch margin
%\usepackage{uarial}
\usepackage{graphicx}
\usepackage{url}
\usepackage{color}
\usepackage{pdfsync}
\usepackage{authblk}
\usepackage{fancyhdr}
\usepackage[numbers, sort, compress]{natbib}
\usepackage{xspace}
\usepackage{mdwlist}
\usepackage{wrapfig}
\usepackage[small, compact]{titlesec}

\usepackage{bibspacing}
%\setlength{\bibspacing}{\baselineskip}

\titlespacing{\section}{0pt}{*0}{*0}
\titlespacing{\subsection}{0pt}{*0}{*0}
\titlespacing{\subsubsection}{0pt}{*0}{*0}

\newcommand{\doc}{Abstractions for D3 Applications and Systems at the Exascale}
\newcommand{\myname}{Jha}

%\pagestyle{fancy}
%\fancyhead{}
%\fancyfoot{}
%\fancyhead[CO,CE]{\lhead{\footnotesize \it \noindent{}\doc{} \hspace{0.5in} \hfill \myname \hfill}}
%\renewcommand\headrulewidth{0.33pt}

%\fancyhead[CO,CE]{\lhead{\footnotesize \hspace{3.0in} Jha Katz Weissman}}
%\fancyfoot[CO,CE]{}
%\fancyfoot[RO, LE] {\thepage}


\usepackage{lscape}
\usepackage{comment}
\setcounter{totalnumber}{50}
\setcounter{topnumber}{50}
\setcounter{bottomnumber}{50}
\parskip = 0.04in

\newcommand{\projectnamefull} {\textit{Abstractions for D3
    Applications and Systems at the Exascale} }

\newcommand{\upup}{\vspace*{-0.5em}}
\newcommand{\up}{\vspace*{-0.25em}}
\newcommand{\I}[1]{\textit{#1}}
\newcommand{\B}[1]{\textbf{#1}}
\newcommand{\T}[1]{\texttt{#1}}
\newcommand{\BI}[1]{\B{\I{#1}}}

\newcommand{\yes}{$\bullet$}

\long\def\finalcomment#1{{\bf \textcolor{red}{\bf #1}}}
\newcommand{\CF}{\finalcomment}

\newif\ifdraft
\drafttrue
\ifdraft
\newcommand{\katznote}[1]{ {\textcolor{magenta} { ***Dan: #1 }}}
%\newcommand{\note}[1]{ {\textcolor{red} { #1 }}}
\newcommand{\jhanote}[1]{ {\textcolor{blue} { ***Shantenu: #1 }}}
\newcommand{\jonnote}[1]{ {\textcolor{red} { ***Jon: #1 }}}
\newcommand{\msnote}[1]{ {\textcolor{green} { ***Mark: #1 }}}
\newcommand{\mtnote}[1]{ {\textcolor{cyan} { ***Matteo: #1 }}}
\else
\newcommand{\katznote}[1]{}
%\newcommand{\note}[1]{}
\newcommand{\jhanote}[1]{}
\newcommand{\jonnote}[1]{}
\newcommand{\msnote}[1]{}
\newcommand{\mtnote}[1]{}
\fi

\hyphenation{Aspects}

\begin{document}
%\maketitle

\vspace{0.25in}
\noindent{\\
{\bf Next Generation Middleware for Distributed Exascale
    Computing Infrastructure}
  \\

\vspace{-0.2in}

\begin{center}
\line(1,0){250}
\end{center}

\vspace{-0.2in}
   % We propose to provide middleware capabilities so as to support
    % both traditional computational science applications, as
    % well as increasingly important and data-intensive DOE
    % experimental facilities such as APS and ITER.

\begin{abstract}
  \footnotesize{Distributed Exascale Computing will emerge as an
    important and synergistic pathway to deliver science at the
    exascale. As an illustrative example, the integration between
    single leadership machines and larger number of less-powerful
    machines is necessary to promote efficient resource utilization
    and support many applications. In order to leverage existing DoE
    investments in leadership-class systems and multi-level solutions
    that integrate adaptive applications and dynamic resource
    management capabilities at extreme-scales, novel middleware
    constructs that go beyond the current state-of-the-art are
    required.  For example, there is a need to move beyond middleware
    that only supports simple interoperable execution of jobs to one
    that has first-class support for well-defined and predictable
    performance and functional properties. Furthermore, middleware
    constructs
    % well-defined {\it capabilities},
%     viz., predictable performance and functional properties, and which
    should enable reasoning about workload distribution -- whether,
    how and when to distribute?
    % \msnote{I can't parse this last
    %   sentence, too many constructs}
    We propose to research, design and prototype middleware for
    distributed exascale computing infrastructure that will support
    current and future DoE Science missions, including but not limited
    to high-energy physics, climate modeling and bioinformatics.}
\end{abstract}


\noindent {\bf Introduction and Scope: }
Although Exascale Computing usually refers to the technology of using
a single tightly-coupled computing resource to deliver performance in
the exaflops region, there are many reasons to believe that
Distributed Exascale Computing (DEC) is an important complementary and
synergistic pathway to deliver science at the exascale.  DEC includes
approaches that require the coupling of exaflop computing to
processing exabytes of data which are often distributed by nature, as
well as the integration between single leadership machines and larger
number of less powerful machines. The importance of DEC has been noted
in DOE reports~\cite{dmav-exa}, as well as the International Exascale
Software Project Roadmap~\cite{iespr}.

%Ref~\cite{iespr} identifies support for external environments as a
%core function of systems management.

Supercomputing class applications as those under development for the
Square Kilometer Array, the Large Synoptic Survey Telescope
(LSST), or for the next generation of combustion, high-energy physics
(e.g. ATLAS) and bioinformatics are becoming increasingly complex,
multi-component, and reliant upon federating distributed data sources
in workflows that span machine size and scales.  In this scenario, the
federation of single leadership machines and larger number of less
powerful machines is necessary to promote efficient resource
utilization.  In addition to computational science applications, DOE
experimental facilities (e.g., Advanced Photon Source) are emerging as
other major source of data (often at rates of hundreds of terabytes
per day).  Scientific progress using these next-generation
experimental facilities is predicated upon efficient distribution of
their data and the ability to support large-scale and often real-time
analysis.  In conjunction with the aforementioned reports and analysis
applications, it can be inferred that scalable capabilities are
required in order to couple exaflops of computing with exabytes of
data. Such a coupling will support novel data analysis and
collaboration therefore fostering scientific discovery.

% \mtnote{The sentence from here on is fairly complex to follow. After
%   three readings I managed to understand the whole of it and I do
%   agree with it. I would unpack it, maybe something on like: ''}

% scalable capabilities that enable flexible modes of coupling of
% exaflops of computing with exabytes of data are required to support
% novel data analysis and collaboration and thereby scientific
% discovery\msnote{Does scalability enable flexibility or the other way
%   around?}.

% In addition, these abilities need to be integrated with the overall
% infrastructure, so as to provide end-to-end data management and
% movement on the one hand, with co-scheduling of analysis and ``main
% computation'' at the other.


\noindent {\bf The Case for Next Generation Middleware for DEC: } The
challenge in DEC is to support flexible execution and federation modes
as well as to provide qualitatively novel abilities with well-defined
and predictable performance and functional properties, which we refer
to as capabilities.  This is not feasible by simply scaling existing
approaches, as current middleware solutions essentially focus on the
implementation of ``individual'' properties (e.g., jobs), and not on
``collective functional properties'' (e.g., number of jobs of
specified functional requirement); this requires a fundamental rethink
of the way we architect capabilities and federate resources.
Specifically, we find that there is a need for middleware that: (i)
enables integrated end-to-end functional properties presented to the
application layer as well-defined capabilities; (ii) provides
first-class support for the coupling of distributed data with compute,
whilst exploiting advances in data-cyberinfrastucture (such as
file-systems, dynamic storage provisioning, reliable data-transfers),
and (iii) is responsive to spatio-temporal variations at
multiple-levels, namely fluctuating resources as well varying
applications properties, i.e. support dynamic resource management and
adaptive application execution.

%\msnote{... dynamic resource and adaptive application execution?}.

% whereby an abstraction of an element of the DEC landscape can be
% modeled in more ways than one, and each of the models in turn can
% have multiple implementations.  Specifically, we will develop both
% Conceptual and Analytical.  (which guide reasoning and
% implementation designs) and analytical models

% \noindent {\bf Project Plan: Middleware for Exascale Distributed
%   Computing Infrastructure: } 

\noindent {\bf Project Plan:} The objective of this project is to
research, design and prototype middleware that will
% provide well-defined
% capabilities % by integrating application workload
% management and resource federation layers
meet requirements (i) - (iii).  To this end, we will define, extend
and refine several {\it conceptual} models that enable reasoning and
help provide semantically consistent behaviour, as well as {\it
  analytical} models that permit quantitative performance analysis. We
will develop models for simplified (skeletal) applications, as well as
models of workload generation and federation of resources.  Our model
of an abstract distributed application (A*), will enable reasoning
about decomposition of an application into well-defined and possibly
correlated tasks. Analytical model (W*) which will be used to generate
quantifiable workloads from these well-defined tasks.  A qualitative
model of infrastructure (I*) will be based upon a set of capability
units exposed by that infrastructure, \I{without} regards to its
internal properties, or specific mechanisms that are used to provide
these capabilities. We will develop a quantitative model (F*) of
federating infrastructure such that they provide well-defined
collective capabilities.  Importantly these models will interface with
each other -- syntactically and semantically, to guide multi-level
reasoning to determine effective workload decomposition, distribution
and placement.  Our work will build upon recent theoretical advances
in models of pilot-abstractions (P*)~\cite{pstar}, which fundamentally
supports and promotes both dynamic compute and data (access, storage,
transfer) provisioning, which in turn builds upon the abstraction of
affinity between distributed compute, data and networking
resources. In summary, we will use an abstractions-based and
model-driven approach to provide well-defined capabilities by an
integrated approach to application workload management and resource
federation layers.

\noindent {\bf Project Impact:} In addition to the DoE relevant
applications alluded to, our middleware developments will support the
execution requirements of novel, high-performance, data-intensive
analysis libraries (think BigData equivalents to PETSc), as well as
provide the distributed and dynamic resource federation capabilities
workflow systems can utilize.  The conceptual and analytical models we
will develop, will offer both qualitative and quantitative insight for
simulating and predicting the end-to-end execution of a range of
scientific workloads in a distributed exascale environment, as well as
be used to provide well-defined middleware capabilities.

% \mtnote{I would consider whether to highlight that we are also
%   proposing a paradigm shift from a notion of 'job' to one of
%   'capability unit' (or something like that). This is relevant as it
%   changes the way we represent a workload and, potentially, also how
%   users at the different levels of the stack conceive an application.}

\footnotesize

\bibliographystyle{unsrt}
\bibliography{doe-ecrp}
\end{document}

